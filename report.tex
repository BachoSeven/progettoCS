\documentclass[a4paper]{article}

\usepackage[T1]{fontenc}
\usepackage{textcomp}
\usepackage[italian]{babel}
\usepackage{hyperref}
\usepackage{amsmath, amssymb, amsthm}
% for \lightning
\usepackage{stmaryrd}
\usepackage{geometry}
\usepackage{tikz-cd}

\hypersetup{
	colorlinks = true, % links instead of boxes
	urlcolor   = cyan, % external hyperlinks
	linkcolor  = blue, % internal links
	citecolor  = cyan   % citations
}

\newcommand{\R}{\mathbb{R}}
\newcommand{\C}{\mathbb{C}}
\newcommand{\Q}{\mathbb{Q}}
\newcommand{\N}{\mathbb{N}}
\newcommand{\A}{\mathbb{A}}
\newcommand{\Z}{\mathbb{Z}}

\renewcommand{\labelitemii}{$\circ$}
\renewcommand{\Im}{\operatorname{Im}}

\newcommand\numberthis{\addtocounter{equation}{1}\tag{\theequation}}

\newtheorem{theorem}{Theorem}[section]
\newtheorem{lemma}{Lemma}[section]

\theoremstyle{definition}
\newtheorem{definition}{Definition}[section]

\theoremstyle{definition}
\newtheorem{example}{Example}[section]

\theoremstyle{remark}
\newtheorem*{remark}{Remark}

\theoremstyle{definition}
\newtheorem{exercise}{Esercizio}[section]
\newtheorem*{exercise*}{Esercizio}

\begin{document}
\section{Analisi della convergenza}
Dall'esecuzione dell'algoritmo, possiamo osservare che, man mano che $\gamma$ si avvicina ad 1, il numero di iterazioni necessarie alla convergenza aumenta rapidamente; questo
è dovuto al fatto che, per come è stata definita la matrice $M$, un valore prossimo ad 1 di $\gamma$ significa che la matrice è più lontana dalla matrice identità, e quindi non ha gli
autovalori concentrati attorno ad 1. Questo non garantisce quindi un buon upper bound per la convergenza di GMRES.
\end{document}
